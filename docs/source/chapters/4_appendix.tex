\appendix
\chapter{Appendix}

\section{Debian Manifesto}
\label{appdx::deb_manifesto}
\noindent
Yang ditulis oleh \textit{Ian A. Murdock} dan direvisi pada 1 Juni 1994.\\
Sumber :
{\scriptsize \url{https://www.debian.org/doc/manuals/project-history/ap-manifesto.en.html}}

\subsection{Apa itu Debian Linux ?}
Debian Linux adalah jenis distribusi linux yang baru. Daripada dikembangkan oleh satu individu atau grup yang terisolasi, seperti dstribusi Linux lainnya yang telah dikembangkan di masa lalu, Debian dikembangkan secara terbuka dengan semangat Linux dan GNU. Tujuan utama dari proyek Debian adalah untuk membuat distribusi yang sesuai dengan nama Linux. Debian sedang ditangani dengan hati-hatidan akan dipelihara dan didukung dengan perawatan serupa.

\noindent
ini juga merupakan usaha untuk menciptakan Distribusi nonkomersial yang dapat bersaing secara efektif di pasar komersial. Pada akhirnya akan didistribusikan oleh \textbf{The Free Software Foundation} dalam bentuk CD-ROM dan \textbf{The Debian Linux Association} akan menawarkan distribusi dalam bentuk Floppy disk (disket) dan Tape beserta panduan dalam format cetak, dukungan teknis dan kebutuhan pengguna lainnya. semua hal diatas hanya akan tersedia dengan hemat biaya, dan selebihnya akan diajukan untuk pengembangan \textit{Free Software} lebih lanjut untuk semua pengguna. Distribusi semacam ini sangat penting bagi keberhasilan sistem operasi Linux di pasar komersial, dan harus dilakukan oleh organisasi dalam posisi untuk keberhasilan sistem operasi linux kedepannya dan merekomendasikan perangkat lunak bebas tanpa adanya tekanan untuk mencari keuntungan lebih.

\subsection{Mengapa Debian dibangun ?}
Distribusi sangat penting untuk masa depan Linux. Intinya, dengan distribusi dapat memberikan kemudahan bagi pengguna yang sebelumnya mencari, mendownload, mengkompilasi, menginstal dan mengintegrasikan sejumlah tool penting yang cukup besar untuk merakit sistem Linux agar dapat bekerja dengan baik, kemudian diganti dengan menitikberatkan beban distribusi kepada konstruksi sistem yang akan ditempatkan pada pencipta distribusi yang karyanya dapat dibagi kepada ribuan pengguna dan hampir semua pengguna Linux dapat merasakan rasa pertama rilis Distribusi Linux dan sebagian besar pengguna akan terus menggunakan distribusi demi kenyamanan bahkan setelah mereka terbiasa dengan system operasi. Dengan demikian, distribusi memang memainkan peranan yang sangat penting.

\noindent
Meskipun sangat penting, distribusi hanya sedikit menarik perhatian pengembang. Ada alasan sederhana untuk ini yaitu "Pengembang tidak mudah dan tidak glamor untuk membangun dan membutuhkan banyak upaya terus-menerus dari sang pembuat untuk menjaga agar distribusi terbebas dari bug dan pembaruan". Ini adalah satu hal yang dapat dilakukan untuk membangun sistem dari nol. Ini cukup lain dari distribusi linux yang sudah ada sebelumnya, Ini dilakukan demi memastikan bahwa sistem mudah dipasang oleh orang lain, dapat diinstall dan dapat digunakan dengan berbagai konfigurasi perangkat keras yang berisi perangkat lunak yang bermanfaat bagi orang lain, dan diperbarui bila komponennya sudah harus diperbarui.

\noindent
Banyak distribusi yang telah dimulai sebagai sistem yang cukup baik, namun seiring berjalannya waktu untuk menjaga distribusi menjadi perhatian sekunder. Kasus dalam hal ini adalah seperti \textit{Softlanding Linux System} yang lebih dikenal dengan \textit{(SLS)}. Ini mungkin distribusi Linux yang paling banyak dikuasai namun dipelihara dengan cara yang buruk. Sayangnya, ini juga mungkin sangat populer. Hal ini, tanpa pertanyaan, distribusi yang menarik banyak perhatian dari "Distributor" Linux komersil yang telah muncul untuk memanfaatkan sistem operasi yang semakin populer.

\noindent
Ini memang kombinasi yang buruk, karena kebanyakan orang yang mendapatkan Linux dari "distributor" dan menerima distribusi Linux yang kacau dan tidak terpelihara dengan baik. Seolah-olah ini tidak cukup buruk, "Distributor" memiliki kecenderungan untuk mengiklankan fitur "produk" yang non-fungsional atau yang sangat tidak stabil dari produk mereka. Dengan fakta bahwa, pembeli tentu saja akan mengharapkan produk itu sesuai dengan iklannya dan kenyataan, dan banyak orang percaya bahwa ini adalah sistem operasi komersil (dan kecenderungan untuk tidak menyebut bahwa Linux itu Bebas atau Distribusi Linux tidak distribusikan di bawah GNU General Public License). Untuk melengkapi semua ini, "Distributor" mendapatkan cukup banyak keuntungan dari usaha mereka membeli iklan yang lebih besar disebagian besar majalah. Ini adalah contoh klasik dari perilaku yang tidak dapat diterima namun dihargai oleh mereka yang sama sekali tidak tahu yang lebih baik. Jelas sesuatu perlu dilakukan untuk memperbaiki situasi ini.

\subsection{Bagaimana cara Debian untuk mengakhiri masalah ini ?}
Proses pengembangan Debian dilakukan secara terbuka untuk memastikan bahwa sistem memiliki kualitas terbaik dan didasari oleh kebutuhan komunitas dan kebutuhan pengguna. Dengan melibatkan orang lain dari berbagai kemampuan dan latar belakang, Debian dapat dikembangkan secara modular. Komponennya berkualias tinggi karena mereka yang memiliki keahlian dibidang tertentu diberi kesempatan untuk membangun atau memelihara komponen secara individu yang terlibat dengan \textit{Debian} diwilayah tertentu. Dan juga melibatkan orang lain untuk mendapatkan saran berharga demi perbaikan system, yang kemudian menjadi rujukan untuk dimasukkan ke dalam distribusi yang sementara dalam proses pengambangan. Dengan demikian, distribusi dibuat berdasarkan kebutuhan dan keinginan pengguna daripada kebutuhan dan keinginan konstruktor. Sangat sulit bagi satu individu atau kelompok kecil untuk mengantisipasi kebutuhan dan keinginan ini tanpa mendapatkan masukan dari orang lain.